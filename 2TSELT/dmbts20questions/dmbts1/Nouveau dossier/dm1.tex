\documentclass[10pt]{article}
\usepackage{luatextra,xcolor}
\usepackage[T1]{fontenc}
\usepackage[utf8]{inputenc}
\usepackage{fourier}
\usepackage[scaled=0.875]{helvet}
\renewcommand{\ttdefault}{lmtt}
\usepackage{amsmath,amssymb,makeidx}
\usepackage{fancybox}
\usepackage{tabularx}
\usepackage[normalem]{ulem}
\usepackage{pifont}
\usepackage{textcomp}
\usepackage{lscape}
\usepackage{graphicx}
\newcommand{\euro}{\eurologo{}}
\usepackage{pst-plot,pstricks-add}
\newtheorem{Ex}{Exemple}
\newtheorem{Exo}{Exercice}
\newcommand{\R}{\mathbb{R}}
\newcommand{\N}{\mathbb{N}}
\newcommand{\D}{\mathbb{D}}
\newcommand{\Z}{\mathbb{Z}}
\newcommand{\Q}{\mathbb{Q}}
\newcommand{\C}{\mathbb{C}}
\setlength{\textheight}{23,5cm}
\setlength{\voffset}{-1,5cm}
\newcommand{\vect}[1]{\mathchoice%
{\overrightarrow{\displaystyle\mathstrut#1\,\,}}%
{\overrightarrow{\textstyle\mathstrut#1\,\,}}%
{\overrightarrow{\scriptstyle\mathstrut#1\,\,}}%
{\overrightarrow{\scriptscriptstyle\mathstrut#1\,\,}}}
\renewcommand{\theenumi}{\textbf{\arabic{enumi}}}
\renewcommand{\labelenumi}{\textbf{\theenumi.}}
\renewcommand{\theenumii}{\textbf{\alph{enumii}}}
\renewcommand{\labelenumii}{\textbf{\theenumii.}}
\def\Oij{$\left(\text{O},~\vect{\imath},~\vect{\jmath}\right)$}
\def\Oijk{$\left(\text{O},~\vect{\imath},~\vect{\jmath},~\vect{k}\right)$}
\def\Ouv{$\left(\text{O},~\vect{u},~\vect{v}\right)$}
\usepackage{fancyhdr}
%\usepackage[dvips]{hyperref}
\usepackage[frenchb]{babel}
\usepackage[np]{numprint}

\newcommand{\mili}[4]{\psgrid[subgriddiv=10, gridlabels=0, gridwidth=0.4pt, subgridwidth=0.4pt,gridcolor=brown!80,subgridcolor=brown!40](#1,#2)(#3,#4)}

\begin{document}
\setlength\parindent{0mm}
\rhead{\textbf{2TSELT}}
\lhead{\small Révision }
\lfoot{\small{Révision}}
\rfoot{\small{Février 2020}}
\renewcommand \footrulewidth{.2pt}
\pagestyle{fancy}
\thispagestyle{empty}

\begin{center}
\vspace{0,5cm}

{\Large \textbf{\decofourleft~Révision : devoir maison de synthèse }}
\end{center}


\begin{enumerate}
\item Calculer la transformée de Laplace de $\cos(\directlua{tex.print(math.random(1,6))}t)\mathcal{U}(t)$.
\item Déterminer l'originale de la fonction $\frac{1}{2}\times \frac{1}{p}-\frac{2}{p+1}+\frac{5}{2}\times \frac{1}{p+2}$
\item Montrer que $\frac{p^{2}+1}{p^{2}+3p+2}=\frac{1}{2}\times \frac{1}{p}-\frac{2}{p+1}+\frac{5}{2}\times \frac{1}{p+2}$
\item Calculer la transformée de Laplace de $y''(t)+3y'(t)+2y(t)$ avec $y(0)=1$ et $y'(0)=0$.
\item On consdère la fonction $f(t)$, paire et $\pi$ périodique telle que :
\begin{align*}
f(t)=\begin{cases} \frac{pi}{2}-t & \text{ si }t\in [0;\frac{\pi}{2}] \\ 0 & \text{ si }t\in [\frac{\pi}{2};\pi] \end{cases}
\end{align*}
Représenter la fonction sur $[-2\pi;2\pi]$.
\item Calculer le coefficient $a_{0}$ de la fonction précédente.
\item Que valent les coefficients $b_{n}$ de la fonction précédente ? Justifier.
\item Résoudre l'équation différentielle $y'(t)+3y(t)=0$.
\item Montrer que $h(t)=t^e{-3t}$ est une solution particulère de $y'(t)+3y(t)=e^{-3t}$.
\item Résoudre l'équation différentielle $y''(t)+2y'(t)+y(t)=0$.
\item Montrer que $h(t)=t^{2}$ est une solution différentielle de $y''(t)+2y'(t)+y(t)=t^{2}+4t+2$.
\item Résoudre l'équation différentielle $y''(t)-8y'(t)+25y(t)=0$.
\item Montrer que $h(t)=2$ est une solution différentielle de $y''(t)-8y'(t)+25y(t)=50$.
\item $X$ suit la loi binomiale $\mathcal{B}(50,0.0\directlua{tex.print(math.random(1,9))})$. Quelle est l'espérance de $X$ ?
\item \begin{luacode}
a=math.random(120,150)
b=math.random(5,10)
tex.print("$X$ suit une loi normale de paramètres $"..a.."$ et $"..b.."$, calculer $P("..a-b.."\\leq X\\leq "..a+b..")$")
\end{luacode}
\item \begin{luacode}
a=math.random(120,150)
b=math.random(11,15)
tex.print("$X$ suit une loi normale de paramètres $"..a.."$ et $"..b.."$, déterminer $h>0$ tel que $P("..a.."-h\\leq X\\leq "..a.."+h)=0.95$")
\end{luacode}
\item  $X$ suit la loi de Poisson de paramètre $\directlua{tex.print(math.random(3,6))})$. Calculer $P(X\ge 7)$.
\item Compléter le tableau suivant et donner les probabilités des événements ainsi que leurs intersections :
\begin{center}
\begin{tabular}{|c|c|c|c|}
\hline 
  &  E  &  $\bar{E}$ & Total\\
\hline
S &     &      $50$      &    $60$  \\
\hline
$\bar{S}$  &    &          &      \\
\hline
 Total &  $100$   &            &  $250$ \\
\hline
\end{tabular}
\end{center}
\item Déterminer $P_{S}(E)$.
\item Déterminer un argument de $1+3\omega$ pour $\omega>0$.
\end{enumerate}


\end{document}